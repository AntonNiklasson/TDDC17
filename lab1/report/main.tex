\documentclass{article}

\usepackage[parfill]{parskip}

\begin{document}

\title{TDDC17: Lab 1}
\author{Henning Hall \& Anton Niklasson}
\maketitle

\section*{Our Solution}
We came up with a solution that concists of three phases:

\begin{enumerate}
	\item Locating corners
	\item Traversing the world
	\item Go back home
\end{enumerate}

The goal of the first phase is to establish some sense of the environment and positioning the agent in a good spot before moving into the traversing.

The first step of the first phase is to move east and south as far as possible. This gives us the first corner. We then move west as far as possible to find the other bottom corner. From this position we initialize the second phase.

The traversing phase moves the agent north while fully scanning each row. This will eventually place us at the top of the world with every bit of dirt cleaned up.

From there all we have to do is go back home. The agent knows its current position and it also remembers the home position.

\section*{Optimizations}

Our solution is probably far from perfect. First of all, by moving the agent around like this we rely heavily on turning around on the spot. This has a cost to it, and that it something that we probably could take more into consideration.

Another point of optimization is to make sure we do not visit a cell more than once. As of right now the agent checks quite a few cells multiple times.

\end{document}