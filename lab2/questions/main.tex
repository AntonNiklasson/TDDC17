\documentclass{article}

\usepackage[parfill]{parskip}

\begin{document}

\title{TDDC17: Lab 2}
\author{Henning Hall \& Anton Niklasson}
\maketitle


\section*{Questions}

\section{In the vacuum cleaner domain in part 1, what were the states and actions? What is the branching factor?}

Answer.

\section{What is the difference between Breadth First Search and Uniform Cost Search in a domain where the cost of each action is 1?}

Answer.

\section{Suppose that h1 and h2 are admissible heuristics (used in for example A*). Which of the following are also admissible?}
\textbf{a) (h1+h2)/2}

Answer.

\textbf{b) 2h1}

Answer.

\textbf{c) max (h1,h2)}

Answer.

\section{If one would use A* to search for a path to one specific square in the vacuum domain, what could the heuristic (h) be? The cost function (g)? Is it an admissible heuristic?}

Answer.

\section{Draw and explain. Choose your three favorite search algorithms and apply them to any problem domain (it might be a good idea to use a domain where you can identify a good heuristic function). Draw the search tree for them, and explain how they proceed in the searching. Also include the memory usage. You can attach a hand-made drawing.}

Answer.

\section{Look at all the offline search algorithms presented in chapter 3 plus A* search. Are they complete? Are they optimal? Explain why!}

Answer.

\section{Assume that you had to go back and do lab 1 once more, but this time with obstacles. Remember that the agent did not have perfect knowledge of the environment but had to explore it incrementally. Could you still use the search algorithms you have learned to guide the agent's execution? What would you search for? Give an example.}

Answer.


\end{document}
